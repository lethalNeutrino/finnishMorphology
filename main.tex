\documentclass{beamer}

\usepackage[utf8]{inputenc}
\usepackage{tipa, vowel, gb4e}
\usepackage{textcomp}

\usetheme{metropolis}
%\usecolortheme{beaver}

\title[Project Presentation] %optional
{Finnish Morphology}

\subtitle{Presented by: Anton Than}

\setbeamertemplate{sections/subsections in toc}[square]

\author{Kaitlyn Du, Rocky Duong, Andrew Liu, Anton Than}

\date {1 November, 2022}
\begin{document}

\renewcommand\thefootnote{\relax}

\frame{\titlepage}

\begin{frame}
	\frametitle{References}
	For this project, we used the papers ``Variation in Allomorph Selection'' by Anttila, ``Exotic Word Formation'' by Bobaljik, ``Finnish Noun Inflection'' by Kiparsky, and ``Finite sentences in Finnish: Word order, morphology, and information structure'' by Nikanne. We also consulted Wikipedia for some examples of Finnish words/sentences/phrases.

	We also consulted with some Finnish speakers for this project. We communicated with a native Finnish speaker, Mette Laine for confirmation of some information we found online. Lastly, many pronunciation guides came from a C1 Finnish learner, August Blackham.
\end{frame}

\begin{frame}
	\frametitle{Intro}
	{\large This presentation is based on the \textbf{Standard Finnish} dialect, spoken mostly in the Southern areas of Finland, and also used by professional speakers of Finnish.}
\end{frame}

\begin{frame}
\frametitle{Table of Contents}
\tableofcontents
\end{frame}

\section{Part 1 --- Morphological Type}

\subsection{Morphological Type}

\begin{frame}
	\frametitle{Morphological Type}
     \begin{center}
         Finnish is primarily a \textbf{synthetic language}, and more specifically an \textbf{agglutinating language}. Some example phrases that demonstrate this are listed below: (Laine 2022).
     \end{center}
     
     \begin{exe}
         \ex
         \gll Puhutteko englanti?\\
         {speak-2.SG-Q} {English}\\
         \glt `Do you speak English?'
         \ex
         \gll K\"aytin tietokonetta.\\
         {use-1.SG-PAST} {information-machine}\\
         \glt `I used a computer.'
     \end{exe}

\end{frame}

\subsection{Examples}

\begin{frame}
	\frametitle{Agglutination Examples}
    Below is a table showing some typical finite verbs in Finnish: (Nikane 2017, p. 70)
    {\scriptsize
    \begin{center}
        \begin{tabular}{|c|p{1cm}|p{1cm}|p{2.5cm}|p{2cm}|}
        \hline
        Stem & (Passive) & Tense/Mood & Subject & Meaning\\\hline
        istu [`sit'] & & -i\newline [PAST] & -mme \newline[1PL SUBJ. AGR] & `we sat down'\\\hline
        istu [`sit'] & & -isi\newline [COND] & -mme \newline[1PL SUBJ. AGR] & `we would sit down'\\\hline
        istu [`sit'] & -tt \newline[passive] & -i\newline [PAST] & -in \newline[PASS SUBJ. AGR] & `it was sat down'\\\hline
        istu [`sit'] & -tta \newline[passive]& -isi\newline [COND] & -in \newline[PASS SUBJ. AGR] & `it would have been sat down'\\\hline
        \end{tabular}
    \end{center}
    }
	
\end{frame}

\begin{frame}
	\frametitle{Agglutination Examples cont.}

    Consider the below Finnish words: (Note: some of these words are not typically used, and are merely an example of what is \textit{theoretically} possible) (Blackham 2022)
 
    \begin{itemize}
        \item istua ``to sit down" (istun ``I sit down")
        \item istahtaa ``to sit down for a while"
        \item istahdan ``I'll sit down for a while"
        \item istahtaisin ``I would sit down for a while"
        \item istahtaisinko ``should I sit down for a while?"
        \item istahtaisinkohan ``I wonder if I should sit down for a while"
        \item istahtaisinkohankaan ``I wonder if I should sit down for a while after all"
    \end{itemize}
    
    Note the separable suffixes: /-ta/, /-dan/, /-isin/, /-ko/, /-han/, /-kaan/
    %should prolly put another slide for this, but im a bit lazy atm, ill do it later
\end{frame}



\section{Part 2 --- Morphological Processes}

\subsection{Verb Conjugation}

\begin{frame}
	\frametitle{Finnish Verb Conjugation}
	Finnish has 6 classes of verbs that conjugate differently, but group 1 suffices to show agglutinative morphology:
	\begin{center}
	\begin{tabular}{c c}
		\hline
		min\"a tied\"an & I know\\
		sin\"a tied\"at & you (singular) know\\
		h\"an/se tiet\"a\"a & (s)he/it knows\\
		se tiet\"a\"a & (s)he/it knows (colloquial)\\
		me tied\"amme & we know\\
		te tied\"atte & you (plural/formal) know\\
		he/ne tiet\"av\"at & they know\\
		ne tiet\"a\"a & they know (colloquial)\\
		\hline
	\end{tabular}
	\end{center}

	From this table, we see that the basic endings for conjugations are -n, -t, -(final vowel), -mme, -tte, and -vat.
\end{frame}

\begin{frame}
	\frametitle{Finnish Verb Conjugation --- Past Tense}
	For the past tense, the infix -i- is inserted between the infinitive and the verb:
	\begin{center}
	\begin{tabular}{c c}
		\hline
		puhun = 'I speak' & puhuin = 'I spoke'\\
		puhut = 'you speak'& puhuit = 'you spoke'\\
		puhuu = '(he) speaks'& puhui = '(he) spoke'\\
		puhumme = 'we speak'& puhuimme = 'we spoke'\\
		\hline
	\end{tabular}
	\end{center}
\end{frame}

\begin{frame}
	\frametitle{Finnish Verb Conjugation --- Passive Mood}
	For the passive mood, there are 4 different endings for present, past, conditional, and potential that all follows the same general construction. The present passive will be used as an example, and its ending is -taan
	\[
		\text{puhua } \rightarrow \text{puhu-} \rightarrow \text{puhutaan}
	.\]
	If the root ends with a or \"a, it is changed to e:
	 \[
		\text{tiet\"a\"a} \rightarrow \text{tied\"a-} \rightarrow \text{tiede-} \rightarrow \text{tiedet\"a\"an}
	.\]
	The endings and examples are listed below:
	{\scriptsize
	\begin{center}
		\begin{tabular}{c c c c}
			\hline
			\hline
			Mood & Ending & Example & Meaning\\
			\hline
			\hline
			Present passive & -taan & puhutaan & `it is spoke'\\
			Past passive & -ttiin & puhuttiin & `it was spoken'\\
			Conditional passive & -ttaisiin & puhuttaisiin & `it would be spoken'\\
			Potential passive & -ttaneen & puhuttaneen & `it may be spoken'\\
			\hline
		\end{tabular}
	\end{center}
	}
\end{frame}


\subsection{Derivational Affixes}

\begin{frame}
	\frametitle{Derivational Affixes}
        Finnish has a lot less free morphemes than English does, so lots of derivational affixes are used to create words: (Laine 2022)
    {\scriptsize
    \begin{center}
        \begin{tabular}{|c|c|p{5cm}|}
            \hline
            \textbf{Suffix} & \textbf{Meaning} & \textbf{Example}\\\hline
            -ja / -j\"a & 
                agents from verbs & 
                lukea ``to read" → lukija ``reader" \\\hline
            -sto / -stö & collective nouns & 
                kirja ``a book" → kirjasto ``a library" \newline 
                laiva ``a ship" → laivasto ``navy, fleet"\\\hline
            -in & 
                instruments or tools & 
                kirjata "to book, to file" → kirjain "a letter" (of the alphabet) \newline
                vatkata "to whisk" → vatkain "a whisk, mixer" \\\hline
            -uri / -yri & 
                agents or instruments &
                kaivaa "to dig" → kaivuri "an excavator" \newline
                laiva "a ship" → laivuri "shipper, shipmaster" \\\hline
        \end{tabular}
    \end{center}
    }
\end{frame}

\begin{frame}
	\frametitle{Derivational Affixes cont.}
    {\scriptsize
    \begin{center}
        \begin{tabular}{|c|p{3cm}|p{5cm}|}
            \hline
            \textbf{Suffix} & \textbf{Meaning} & \textbf{Example}\\\hline
            -uri / -yri &
                agents or instruments &
                kaivaa "to dig" → kaivuri "an excavator" \newline
                laiva "a ship" → laivuri "shipper, shipmaster" \\\hline
            -os / -ös &
                result nouns from verbs &
                tulla "to come" → tulos "result, outcome" \newline
                tehd\"a "to do" → teos "a piece of work" \\\hline
            -ton / -tön &
                adjectives indicating the lack of something &
                onni "happiness" → onneton "unhappy" \newline
                koti "home" → koditon "homeless" \\\hline
            -kas / -k\"as &
                adjectives from nouns &
                itse "self" → itsek\"as "selfish" \newline
                neuvo "advice" → neuvokas "resourceful" \\\hline
            -va / -v\"a &
                adjectives from verbs &
                taitaa "to be able" → taitava "skillful" \newline
                johtaa "to lead" → johtava "leading" \\\hline
        \end{tabular}
    \end{center}
    }
\end{frame}

\begin{frame}
	\frametitle{Derivational Affixes cont.}
    {\scriptsize
    \begin{center}
        \begin{tabular}{|c|p{3cm}|p{5cm}|}
            \hline
            \textbf{Suffix} & \textbf{Meaning} & \textbf{Example}\\\hline
            -llinen &
                adjectives from nouns &
                lapsi "child" → lapsellinen "childish" \newline
                kauppa "a shop, commerce" → kaupallinen "commercial" \\\hline
            -la / -l\"a &
                locations (places related to the stem) &
                kana "a hen" → kanala "a henhouse" \newline
                pappi "a priest" → pappila "a parsonage" \\\hline
            -lainen / -l\"ainen &
                inhabitants (of places), among others &
                Englanti "England" → englantilainen "English person/thing" \newline
                Ven\"aj\"a "Russia" → ven\"al\"ainen "Russian person or thing". \\\hline
        \end{tabular}
    \end{center}
    }
    \textbf{Note.} To choose which of the two suffixes is used, refer to the vowel harmony rules from our phonology project. For many Finnish suffixes involving vowels, there will be two such variations based on those rules.
\end{frame}

\subsection{Non-concatenative Process: the Illative Case}
\begin{frame}
  \frametitle{Non-concatenative Process: the Illative Case}
    The illative case in Finnish demonstrates root-and-pattern morphology. (Bobaljik 2002, p. 15)
    \begin{center}
        \begin{tabular}{ c c }
            \hline\hline
            \textbf{Word} & \textbf{Illative Case}\\\hline\hline
            auto `car' &
                autoon `to the car' \\
            koulu `school' &
                kouluun `to school' \\
            p\"aiv\"a `day' &
                p\"aiv\"a\"an `to the day' \\
            kuva `picture' &
                kuvaan `to the picture' \\
            bussi `bus' &
                bussiin `to the bus' \\\hline
        \end{tabular}
    \end{center}
    Note that given the root, the pattern to form the illative case is simply taking the last vowel of the root, then adding the /-n/ suffix.
\end{frame}


\subsection{Allomorphs of the Nominalization Suffix}

\begin{frame}
	\frametitle{Finnish Nominalization}
    In Finnish, the nominalization suffix has three allomorphs: /-nti/, /-nta/, and /-nto/. Some uses are listed below:  (Anttila 2000, p. 37)
    {\scriptsize
    \begin{center}
        \begin{tabular}{c c c}
        \hline
            \textipa{j\'uo-n.ti} & `drink-nom` & `drinking' \\
            \textipa{s\'i.jai-n.ti} & `locate-nom` & `location`\\
            \textipa{\'ar.vi.\`oi-n.ti} & `estimate-nom` & 'estimation'\\
            \textipa{f\'or.ma.li.s\`oi-n.ti}& `formalize-nom` & 'formalization'\\
            \textipa{l\'uo-n.to} & `create-nom` & 'nature'\\
            \textipa{p\'yy-n.t\"o} &`request-nom` & 'request'\\
            \textipa{l\'as.ke-n.to} & 'count-nom' & 'elementary artihmetic'\\
            \textipa{l\'u.e-n.to} & 'read-nom' & 'lecture'\\
            \textipa{\'a.su-n.to} & 'inhabit-nom' & 'apartment'\\
            \textipa{\'us.ko-n.to} & 'believe-nom' & 'religion'\\
            \textipa{l\'u.e-n.ta} & 'read-nom' & 'reading'\\
            \textipa{l\'s.ke-n.ta} & 'count-nom' & 'counting'\\
            \textipa{\'an.saj-n.ta} & 'earn-nom' & 'earning'\\
            \textipa{p\'a.hek.s\`u-n.ta} & 'disapprove-nom' & 'disapproval'\\
            \textipa{\'e.leh.d\`i-n-"a} & 'gesture-nom' & 'gesticulation'\\
            \textipa{v\'e.ti.t\'eh.di-n.t\"a} & 'loiter-nom' & 'loitering'\\\hline
        \end{tabular}
    \end{center}
    }
\end{frame}

\begin{frame}
	\frametitle{Allophones of Nominalization Suffix: /-nta/, /-nti/}
    Out of the three suffixes, two are phonologically conditions on stress (or prosodically conditioned): /-nti/ and /-nta/. /-nti/ follows light syllables, whereas /-nta/ follows heavy syllables. (Anttila 2000, p. 37)
    {\scriptsize
    \begin{center}
        \begin{tabular}{c c c}
        \hline
            \textipa{j\'uo-n.ti} & `drink-nom` & `drinking' \\
            \textipa{s\'i.jai-n.ti} & `locate-nom` & `location`\\
            \textipa{\'ar.vi.\`oi-n.ti} & `estimate-nom` & 'estimation'\\
            \textipa{f\'or.ma.li.s\`oi-n.ti}& `formalize-nom` & 'formalization'\\
            \textipa{l\'u.e-n.ta} & 'read-nom' & 'reading'\\
            \textipa{l\'as.ke-n.ta} & 'count-nom' & 'counting'\\
            \textipa{\'an.saj-n.ta} & 'earn-nom' & 'earning'\\
            \textipa{p\'a.hek.s\`u-n.ta} & 'disapprove-nom' & 'disapproval'\\
            \textipa{\'e.leh.d\`i-n-"a} & 'gesture-nom' & 'gesticulation'\\
            \textipa{v\'e.ti.t\'eh.di-n.t\"a} & 'loiter-nom' & 'loitering'\\\hline
        \end{tabular}
    \end{center}
    }
\end{frame}

\begin{frame}
	\frametitle{Allophones of Nominalization Suffix: /-nta/, /-nti/ cont.}
    Thus, the following rule describes the usage of the two phonologically conditioned suffixes:
    \[
    \text{/-{\small NOM}/} \rightarrow
    \begin{cases}
        \text{[-nti] /} & \text{light syllable \_ }\\
        \text{[-nta] /} & \text{heavy syllable \_ }
    \end{cases}
    \]
\end{frame}

\begin{frame}
	\frametitle{Allophones of Nominalization Suffix: /-nto/}
    {\small ``The suffix /-nto/ is different in two ways: it does not appear to have any prosodic limitations, but freely occurs after both heavies and lights, and it is clearly lexicalized: besides being unproductive, it is typically associated with unpredictable meanings.''} (Anttila 2000, p. 37)

    {\scriptsize
    \begin{center}
        \begin{tabular}{c c c}
        \hline
            \textipa{l\'uo-n.to} & `create-nom` & 'nature'\\
            \textipa{p\'yy-n.t\"o} &`request-nom` & 'request'\\
            \textipa{l\'as.ke-n.to} & 'count-nom' & 'elementary artihmetic'\\
            \textipa{l\'u.e-n.to} & 'read-nom' & 'lecture'\\
            \textipa{\'a.su-n.to} & 'inhabit-nom' & 'apartment'\\
            \textipa{\'us.ko-n.to} & 'believe-nom' & 'religion'\\\hline
        \end{tabular}
    \end{center}
    }
\end{frame}

\begin{frame}
	\frametitle{Allophones of Nominalization Suffix: /-nto/}
    As a result of how /-nto/ is very different from the other two nominalization suffixes, many verbs take both one of /-nta/ or /-nti/ \textit{and} /-nto/, creating two different words:

    {\scriptsize
    \begin{center}
        \begin{tabular}{c c c c c}
        \hline\hline
            \multicolumn{2}{c}{Transparent} & & \multicolumn{2}{c}{Lexicalized}\\\hline\hline
            \textipa{l\'uo-n.ti} & 'creating' 
                & & \textipa{l\'uo-n.to}  & 'nature' \\
            \textipa{l\'ue-nta}  & 'reading' 
                & & \textipa{l\'ue-nto}  & 'lecture'\\
            \textipa{l\'aske-nta}  & 'counting' 
                & &\textipa{l\'aske-nto}  & 'elementary arithmetic'\\
            \textipa{h\'alli-nta}  & 'governing' 
                & & \textipa{h\'alli-nto} & 'government'\\
            \textipa{\'istu-nta}  & 'sitting' 
                & & \textipa{\'istu-nto} & 'session'\\
            \textipa{k\'ukj-nta}  & 'flowering' 
                & & \textipa{k\'ukj-nto} & 'blossom'\\
            \textipa{p\'alki-nta}  & 'rewarding' 
                & & \textipa{p\'alki-nto} & 'prize'\\\hline
        \end{tabular}
    \end{center}
    }
\end{frame}




\section{Sources}

\begin{frame}
	\frametitle{Citations}
	\setbeamertemplate{bibliography item}[text]
	\begin{thebibliography}{10}
		\bibitem[1]{Antilathehun} Anttila, A. (2000). Variation in Allomorph Selection. Proceedings of the North East Linguistic Society, 30(4). https://doi.org/https://scholarworks.umass.edu/nels/vol30/iss1/4 
		\bibitem[2]{Boba} Bobaljik, J. D. (2002). Exotic Word Formation. Morphology. 
		\bibitem[3]{Kiparsky} Kiparsky, P. (2003). Finnish Noun Inflection. Generative Approaches to Finnic and Saami Linguistics, 109–161. 
		\bibitem[4]{neko} Nikanne, U. (2018). Finite sentences in Finnish: Word order, morphology, and information structure. I L. R. Bailey, \& M. Sheehan (Red.), Order and structure in syntax I: Word order and syntactic structure (s. 69–97). Language science press. https://doi.org/10.5281/zenodo.1117686
	\end{thebibliography}
\end{frame}

\begin{frame}
	\frametitle{Citations --- People}
	\setbeamertemplate{bibliography item}[text]
	\begin{thebibliography}{2}
		\bibitem{Blackham} Blackham, August, Personal Communication. (October 27, 2022).
		\bibitem{Laine} Laine, Mette, Personal Communication. (October 27, 2022).
	\end{thebibliography}
\end{frame}

\end{document}
